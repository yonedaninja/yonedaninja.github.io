\documentclass{amsart}

\usepackage{amsmath,amsfonts,amssymb}

\usepackage{titling}
\newcommand{\subtitle}[1]{%
  \posttitle{%
    \par\end{center}
    \begin{center}\large#1\end{center}
    \vskip0.5em}%
}

\usepackage[all]{xy}

\usepackage{tikz-cd}

\usepackage[utf8]{inputenc}
\def\V{\mathcal{V}}
\def\VCat{\V\text{-}\mathbf{Cat}}
\def\Cat{\mathbf{Cat}}
\def\Ban{\mathbf{Ban}}
\def\Lan{\text{Lan}}

\title{Esercizi 2-categorie}
\subtitle{Secondo foglio: cofini, estensioni di Kan, limiti pesati}
\author{Fosco Loregian}

\begin{document}
\maketitle
\begin{itemize}
\item Dimostrare che ogni volta che esiste una aggiunzione parametrica
\[
\mathcal A(F_A(X),Y)\cong \mathcal B(X, G_A(Y))
\]
per due funtori $F : \mathcal A\times\mathcal X\to\mathcal B,G : \mathcal B\times\mathcal A^\text{op}\to\mathcal B$, allora l'unità $\eta : 1 \Rightarrow G_AF_A$ è un cuneo in $A$, e la counità un cocuneo. Sono anche universali?
\item Dimostrare che l'insieme delle trasformazioni naturali $F\Rightarrow G$ è l'equalizzatore
\[
\xymatrix{
\text{Nat}(F,G) \ar@{.>}[r]& \prod_{C\in\mathcal C} \mathcal D(FC,GC) \ar@<4pt>[r]^\alpha\ar@<-4pt>[r]_\beta & \prod_{f : C\to C'} \mathcal D(FC,GC') 
}
\]
per opportune mappe $\alpha,\beta$.
\item Mostrare che il limite di $F : \mathcal A \to \mathbf{Set}$ pesato da $G : \mathcal A \to \mathbf{Set}$ è l'insieme delle trasformazioni naturali $F\Rightarrow G$.
\item Dimostrare il teorema di Brouwer (per assurdo, se esiste una retrazione del disco sulla sfera\dots).
\item Dimostrare che $\Lan_{GG'}\cong \Lan_G\circ\Lan_{G'}$ per due funtori componibili $G',G$.
\item Se $\mathcal C = \mathbf{Vect}$ è la categoria degli spazi vettoriali, mostrare che il funtore $V\mapsto \int^W W^* \otimes V \otimes W$ è la parte sugli oggetti di una monade su $\mathcal C$.
\item L'\emph{oggetto comma} di un diagramma $X \xrightarrow{f}Z\xleftarrow{g}Y$ di categorie è un oggetto $X \xleftarrow{p}(f/g)\xrightarrow{q}Y$ terminale con una trasformazione naturale $fp\Rightarrow gq$. Trovare un peso $W : \{0\to 1\leftarrow 2\} \to \mathbf{Cat}$ per cui $(f/g)\cong \lim{}^WF$, se $F$ è il diagramma $X \xrightarrow{f}Z\xleftarrow{g}Y$.
\item Mostrare che
\[\{W,F\} \cong \int_A \{WA,FA\} \qquad\qquad W\odot F \cong \int^A WA \odot FA\]
\item Chi è il limite pesato di $F : \mathcal A \to \mathcal B$ lungo il funtore $W : \mathcal A\to \mathcal V$ che è costante in un oggetto $W\in\mathcal V$?
\item Mostrare che dato un funtore $F : \mathcal A\to \mathbf{Set}$ esiste un'aggiunzione $\Lan_yF\dashv\Lan_Fy$ dove $y : \mathcal A \to [\mathcal A^\text{op},\mathbf{Set}]$ è l'embedding di Yoneda; mostrare se se $F$ preserva i limiti finiti, lo stesso fa $\Lan_yF$. Il funtore $\Lan_yF$ si chiama \emph{realizzazione} di $F$, e il funtore $\Lan_Fy$ si chiama $F$-nervo.
\item Se $R$ è un anello commutativo, $M\otimes_R N$ è la cofine di una opportuna coppia di funtori $\bar M,\bar N$.
\item Generalizzare l'aggiunzione tra realizzazione e $F$-nervo al caso di un funtore \emph{multilineare}: dato $F \colon \mathcal C_1 \times \dots \times \mathcal C_n \to \mathbf{Set}$, dove ogni  $\mathcal{C}_i$ è piccola, mostrare che esiste un'equivalenza di categorie
\[
\Cat(\mathcal{C}_1 \times \dots \times \mathcal{C}_n , \mathbf{Set}) \cong 
\textsf{Mult}(\widehat{\mathcal{C}_1}\times \dots \times \widehat{\mathcal{C}_n}, \mathbf{Set})
\]
dove $\textsf{Mult}(\_\, , \_\,)$ è la categoriadei funtori cocontinui in ogni variabile una volta che tutte le altre sono state fissate (lo si dimostri per induzione, componendo successive estensioni di Kan). Data $\theta \in \Cat(\mathcal{C}_1 \times \dots \times \mathcal{C}_n , \mathbf{Set})$, descrivere l'aggiunto destro di ciascun $\theta(c_1,\dots,c_i^\circ,\dots, c_n)\colon \widehat{\mathcal{C}_i} \to \mathbf{Set}$ ($c_i^\circ$ significa che tutti gli oggetti $c_j$ sono fissi per $j\neq i$ e $c_i\in\mathcal{C}$ è libero di variare). Tutti questi funtori hanno un `nervo vettoriale' $N\colon \mathbf{Set} \to \widehat{\mathcal{C}_1}\times \dots \times \widehat{\mathcal{C}_n}$.
\end{itemize}

\vspace{\fill}
\begin{flushright}
\today,\\
Fosco Loregian
\end{flushright}
\end{document}