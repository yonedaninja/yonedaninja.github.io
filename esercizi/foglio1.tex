\documentclass[a4paper,10pt]{amsart}

\usepackage{amsmath,amsfonts,amssymb}

\usepackage{titling}
\newcommand{\subtitle}[1]{%
  \posttitle{%
    \par\end{center}
    \begin{center}\large#1\end{center}
    \vskip0.5em}%
}

\usepackage[all]{xy}

\usepackage{tikz-cd}

\usepackage[utf8]{inputenc}
\def\V{\mathcal{V}}
\def\VCat{\V\text{-}\mathbf{Cat}}
\def\Cat{\mathbf{Cat}}
\def\Ban{\mathbf{Ban}}

\title{Esercizi 2-categorie}
\subtitle{Primo foglio: categorie arricchite}
\author{Fosco Loregian}

\begin{document}
\maketitle
\begin{itemize}
	\item Una categoria monoidale è l'ambiente adeguato per definire \emph{oggetti monoide}: un oggetto monoide in $\V$ è un oggetto $M$ con una mappa $\mu : M\otimes M\to M$ che sia associativa, e una mappa $\eta : I\to M$ che faccia da unità per $\mu$: ciò significa che i diagrammi
	\[\xymatrix{
	M\otimes M\otimes M \ar[r]^{M\otimes\mu}\ar[d]_{\mu\otimes M}& M\otimes M \ar[d]^\mu\\
	M\otimes M \ar[r]_\mu & M
	}\qquad 
	\xymatrix{
	I\otimes M\ar[dr]_l \ar[r]^{\eta\otimes M}& M\otimes M \ar[d]^\mu & M\otimes I \ar[l]_{M\otimes \eta}\ar[dl]^r \\
	& M &
	}\] siano commutativi. Determinate cosa è un monoide interno alla categoria degli insiemi; determinate cos'è un monoide interno alla categoria degli spazi vettoriali; determinate cos'è un monoide interno alla categoria $\Cat$ delle categorie e funtori.
	\item Mostrare esplicitamente che $(\mathbf{Set}_*,\lor)$ sottintende una struttura monoidale sulla categoria degli insiemi puntati; mostrare esplicitamente le regole di coerenza per associatore e unitore. Stessa domanda per la categoria degli \emph{spazi topologici} puntati. E' vero che $A\lor -$ commuta coi colimiti?
	\item Definiamo il \emph{prodotto schiacciato} di due insiemi puntati come il pushout
	\[\xymatrix{A\lor B \ar[r]\ar[d]& A\times B\ar[d] \\ {*}\ar[r] & A\land B}\]
	$(A,B)\mapsto A\land B$ è una struttura monoidale su $\mathbf{Set}_*$? Stessa domanda per la categoria degli spazi topologici puntati. E' vero che $A\land -$ commuta coi colimiti?
	\item Dimostrare che la categoria $\text{Fin}_*$ degli insiemi finiti puntati $\{*,1,\dots,n\}$ e funzioni puntate (le $f : [m]_*\to [n]_*$ tali che $f(*_{[m]})=*_{[n]}$) è equivalente alla categoria degli insiemi finiti e funzioni parziali. Che cosa diventa la struttura monoidale $(\text{Fin}_*,\lor)$ lungo questa equivalenza? Che cosa diventa la struttura monoidale $(\text{Fin}_*,\land$ lungo questa equivalenza?
	\item Se $F : \V \leftrightarrows \mathcal{W} : G$ è una coppia di funtori aggiunti monoidali, mostrare o confutare che
	\begin{itemize}
		\item L'unità $\alpha : 1\Rightarrow GF$ e la counità $\epsilon : FG \Rightarrow 1$ sono trasformazioni naturali monoidali;
		\item Il ``cambio di base mediante $F$'', $F_* : \VCat \to \mathcal{W}\text{-}\Cat$ ha per aggiunto destro il cambio di base mediante $G$.
	\end{itemize}
	\item Dimostrare che la sottocategoria $\mathcal S = \{\varnothing, 1\}\subset\mathbf{Set}$ è cartesiana chiusa; è vero o falso che un insieme coincide con una $\mathcal S$-categoria?
	\item La categoria dei gruppi è cartesiana chiusa? Possiede una struttura monoidale $\otimes$ per cui è chiusa?
	\item Sia $G$ un gruppo fissato; nella categoria $\mathbf{Set}^G$ degli insiemi dotati di una azione di $G$ è possibile definire un bifuntore $\otimes_G$ come
	\[X\otimes_G Y := \text{coeq}\left( \xymatrix{\coprod_{g\in G} X\times Y \ar@<4pt>[r]^-{1\times g}\ar@<-4pt>[r]_-{g\times 1} & X\times Y}\right)
	\]
	Questo (insieme all'insieme terminale con azione banale, e alle ovvie coerenze) definisce una struttura monoidale su $\mathbf{Set}^G$?
	\item Dimostrare che per ogni categoria piccola $A$, la categoria dei funtori $[A,\mathbf{Set}]$ è cartesiana chiusa; dedurne che la categoria degli insiemi con una azione di gruppo è cartesiana chiusa. 
	\item Mostrare la \emph{regola di interscambio} per trasformazioni $\V$-naturali:
	\[(\alpha\circ\beta)\bullet(\gamma\circ\delta) = (\alpha\bullet\gamma)\circ(\beta\bullet\delta).\]
	\item Dimostrare che se esiste una aggiunzione $F : \mathcal V\leftrightarrows \mathcal W :  G$ ($F : \mathcal V\to \mathcal W$ aggiunto sinistro), allora esiste una aggiunzione $F_\dag\dashv G_\dag$.
	\item Un'equivalenza di categorie è sempre un funtore monoidale forte? Se $F\dashv G$ è un'equivalenza monoidale, dimostrare che $F_\dag\dashv G_\dag$ induce un'equivalenza tra le $\mathcal V$-categorie e le $\mathcal W$-categorie: restano indotti degli isomorfismi di categorie $\mathcal V\text{-}\mathbf{Cat}[A,A']\cong \mathcal W\text{-}\mathbf{Cat}[F_\dag A, F_\dag A']$.
	\item Data una categoria monoidale $\V$, mostrare che esiste una categoria $\V^\otimes$ così definita:
	\begin{itemize}
		\item gli oggetti di $\V^\otimes$ sono $n$-uple di oggetti di $\V$ denotate $[C_1,\dots,C_n]$ (con la convenzione che se $n=0$ la tupla è vuota);
		\item i morfismi $[C_1,\dots, C_n]\to [D_1,\dots,D_m]$ sono coppie $(\alpha,\{f_j\})$ dove $\alpha$ è una funzione parziale $[n]\to [m]$ con dominio $S_\alpha$ e $\{f_j : \bigotimes_{\{i\mid \alpha(i)=j\}} C_i\to D_j \mid 1\le j\le m\}$ è una famiglia di morfismi di $\V$.
	\end{itemize}
	(definire opportunamente la composizione di due morfismi $(\alpha,f), (\beta,g)$). 
	\begin{itemize}
		\item Mostrare che esiste un funtore $p : \V^\otimes\to \text{Fin}_*$ che manda $[C_1,\dots,C_n]$ in $[n]_*$; mostrare che $p$ è una \emph{opfibrazione}: per ogni oggetto $\overline C= [C_1,\dots,C_n]\in\V^\otimes$ e ogni morfismo $f : [n]_* \to [m]_*$ in $\text{Fin}_*$ esiste un morfismo $(\theta_f,\bar f) : \overline C \to \overline D = [D_1,\dots, D_m]$ tale che $p(\theta_f,\bar f)=f$, e tale che la composizione con $\bar f$ induce la seguente biiezione per ogni $\overline E = [E_1,\dots,E_d]$ 
		\[\V^\otimes(\overline D, \overline E) \cong \V^\otimes(\overline C, \overline E) \times_{\text{Fin}_*([n]_*, [d]_*)}\text{Fin}_*([m]_*, [d]_*).\]
		\item Mostrare che se indichiamo $\V^\otimes_n$ la fibra di $[n]_*$ mediante $p$, $p$ induce un funtore $\V^\otimes_m \to \V^\otimes_n$ per ogni $f : [m]_* \to [n]_*$ in $\text{Fin}_*$.
		\item Mostrare che $\V^\otimes_0 \cong \{0\}$, $\V^\otimes_1 \cong \V$, e in generale che $\V^\otimes_n \cong \V\times\dots\times\V$ ($n$ volte).
		\item Mostrare che la corrispondenza che manda $\V$ in $\V^\otimes$ è funtoriale in $\Cat$; 
		\item Mostrare che $\text{Fin}_* \cong \{0\}^\otimes$ rispetto all'unica struttura monoidale che esiste sulla categoria terminale $\{0\}$; mostrare che $p$ è il funtore indotto dall'unico funtore terminale $\V \to \{0\}$.
	\end{itemize}
	\[  \]
\end{itemize}

\vspace{\fill}
\begin{flushright}
\today,\\
Fosco Loregian
\end{flushright}
\end{document}